% Options for packages loaded elsewhere
\PassOptionsToPackage{unicode}{hyperref}
\PassOptionsToPackage{hyphens}{url}
\documentclass[
  11pt,
]{article}
\usepackage{xcolor}
\usepackage[margin=0.75in]{geometry}
\usepackage{amsmath,amssymb}
\setcounter{secnumdepth}{-\maxdimen} % remove section numbering
\usepackage{iftex}
\ifPDFTeX
  \usepackage[T1]{fontenc}
  \usepackage[utf8]{inputenc}
  \usepackage{textcomp} % provide euro and other symbols
\else % if luatex or xetex
  \usepackage{unicode-math} % this also loads fontspec
  \defaultfontfeatures{Scale=MatchLowercase}
  \defaultfontfeatures[\rmfamily]{Ligatures=TeX,Scale=1}
\fi
\usepackage{lmodern}
\ifPDFTeX\else
  % xetex/luatex font selection
\fi
% Use upquote if available, for straight quotes in verbatim environments
\IfFileExists{upquote.sty}{\usepackage{upquote}}{}
\IfFileExists{microtype.sty}{% use microtype if available
  \usepackage[]{microtype}
  \UseMicrotypeSet[protrusion]{basicmath} % disable protrusion for tt fonts
}{}
\makeatletter
\@ifundefined{KOMAClassName}{% if non-KOMA class
  \IfFileExists{parskip.sty}{%
    \usepackage{parskip}
  }{% else
    \setlength{\parindent}{0pt}
    \setlength{\parskip}{6pt plus 2pt minus 1pt}}
}{% if KOMA class
  \KOMAoptions{parskip=half}}
\makeatother
\usepackage{graphicx}
\makeatletter
\newsavebox\pandoc@box
\newcommand*\pandocbounded[1]{% scales image to fit in text height/width
  \sbox\pandoc@box{#1}%
  \Gscale@div\@tempa{\textheight}{\dimexpr\ht\pandoc@box+\dp\pandoc@box\relax}%
  \Gscale@div\@tempb{\linewidth}{\wd\pandoc@box}%
  \ifdim\@tempb\p@<\@tempa\p@\let\@tempa\@tempb\fi% select the smaller of both
  \ifdim\@tempa\p@<\p@\scalebox{\@tempa}{\usebox\pandoc@box}%
  \else\usebox{\pandoc@box}%
  \fi%
}
% Set default figure placement to htbp
\def\fps@figure{htbp}
\makeatother
\setlength{\emergencystretch}{3em} % prevent overfull lines
\providecommand{\tightlist}{%
  \setlength{\itemsep}{0pt}\setlength{\parskip}{0pt}}
\usepackage{bookmark}
\IfFileExists{xurl.sty}{\usepackage{xurl}}{} % add URL line breaks if available
\urlstyle{same}
\hypersetup{
  pdftitle={The Application of Machine Learning on Risk Classification of Heart Disease},
  pdfauthor={Xiaoyu Lin; Kyoka Ono},
  hidelinks,
  pdfcreator={LaTeX via pandoc}}

\title{The Application of Machine Learning on Risk Classification of
Heart Disease}
\author{Xiaoyu Lin \and Kyoka Ono}
\date{}

\begin{document}
\maketitle

GitHub Link:
\url{https://github.com/kyokaono8811/biostat625finalproject.git}

Contributions:

\begin{itemize}
\tightlist
\item
  Xiaoyu Lin --- ``Data description, Data cleaning\ldots{}''\\
\item
  Kyoka Ono --- ``\ldots{}''
\end{itemize}

\section{Abstract}\label{abstract}

Heart disease is a leading cause of death for adults in the U.S, and
early detection of key risk factors is essential for prevention. In this
study, we apply multiple machine learning methods, including logistic
regression, random forest, GAM, ikNN, XGBoost, and RNN identify the
significant risk factors of heart disease. We then evaluate model
performance and compare variable importance across methods using
evaluation Metrics.

(Some results conclusion)

\section{Introduction}\label{introduction}

Heart disease affects millions of individuals in the U.S, making early
identification of risk factors a high public-health priority.
Traditional statistical approaches have identified several predictors,
but machine learning methods can capture more complex, nonlinear
relationships.

This project applies several different machine learning algorithms to
the \texttt{heart\_2022\_no\_nans.csv} dataset from the CDC Behavioral
Risk Factor Surveillance System (BRFSS) on Kaggle. Our goal is to
determine which model outputs the highest evaluation metric for
predicting heart attack, and which variables significantly associated
with having had a heart attack?

\section{Methods}\label{methods}

\subsubsection{\texorpdfstring{\textbf{Data
Source}}{Data Source}}\label{data-source}

The dataset contains 40 variables for over 200,000 survey participants
(all complete cases).

\begin{verbatim}

Number of participants: 246022
\end{verbatim}

\begin{verbatim}
Number of variables: 40
\end{verbatim}

\subsubsection{\texorpdfstring{\textbf{Preprocessing}}{Preprocessing}}\label{preprocessing}

\textbf{Variable Selection}

We choose 10 covariates out of 40 for our models based on literature
review and use \texttt{HadHeartAttack} as the predicted variable

Outcome Variable:

\begin{itemize}
\tightlist
\item
  \texttt{HadHeartAttack}: Binary indicator (Yes/No) of whether a doctor
  diagnosed the respondent with a heart attack.
\end{itemize}

Predictor Variables:

\begin{itemize}
\item
  \texttt{Sex}: Biological sex of the participant (Male/Female).
\item
  \texttt{PhysicalActivities}: Whether the participant engaged in
  physical activities in the past month (Yes/No).
\item
  \texttt{SleepHours}: Average number of hours of sleep per night
  (numeric).
\item
  \texttt{HadStroke}: Whether had a stroke (Yes/No).
\item
  \texttt{HadDiabetes}: Whether had a diabetes (Yes/No/Yes, but only
  during pregnancy (female)/No, pre-diabetes or borderline diabetes).
\item
  \texttt{SmokerStatus}: Smoking status of the participant (Former
  smoker/Never smoked/Current smoker -- now smokes every day/Current
  smoker -- now smokes some days/No).
\item
  \texttt{RaceEthnicityCategory}: (White only, Non-Hispanic/Black only,
  Non-Hispanic/Other race only, Non-Hispanic/Multiracial,
  Non-Hispanic/Hispanic)
\item
  \texttt{AgeCategory}: Age group (18--24, 25--29, \ldots, 80+).
\item
  \texttt{BMI}: Body mass index (numeric).
\item
  \texttt{AlcoholDrinkers}: Whether a participant is a heavy drinkers
  (adult men having more than 14 drinks per week and adult women having
  more than 7 drinks per (Yes/No).
\end{itemize}

\textbf{Data Cleaning}

We would like to create dummy variables for categorical variables and
merge equivalent levels.

\subsubsection{\texorpdfstring{\textbf{Data
Description}}{Data Description}}\label{data-description}

Let's look at the distribution of the variables

\pandocbounded{\includegraphics[keepaspectratio]{final_report_files/figure-latex/unnamed-chunk-7-1.pdf}}

From the graph above, we can see that some categorical variables such as
\texttt{HadHeartAttack}, \texttt{HadStroke}, and
\texttt{RaceEthnicityCategory} exhibit noticeable class imbalance. Other
categorical variables are more balanced. Numerical variables like
\texttt{SleepHours}, and \texttt{BMI} show skewed distributions.

Now Let us build a correlation table. We use spearman correlation
because the dataset contains a mixture of categorical and numerical
variables. Spearman does not assume linearity or normality, making it a
better measure of association than Pearson for this dataset.

\pandocbounded{\includegraphics[keepaspectratio]{final_report_files/figure-latex/unnamed-chunk-8-3.pdf}}

All pairwise correlations were relatively weak, indicating that no
individual predictor shows a strong linear association with heart attack
or with each other. This is expected in multi-factor health datasets,
where the outcome is influenced by many small effects rather than a
single dominant variable.

Low pairwise correlations do not imply weak predictive power for
nonlinear effects, interactions, and combined contributions can still
provide meaningful classification performance in multivariate models.

\subsubsection{\texorpdfstring{\textbf{Models
Applied}}{Models Applied}}\label{models-applied}

Highly imbalanced classes may affect model performance if not addressed,
so we will use the method of undersampling or class weights during model
training. And to improve model performance, the skewed numerical
variables should be standardized or centered, especially for algorithms
sensitive to scale (e.g.~logistic regression).

\begin{enumerate}
\def\labelenumi{\arabic{enumi}.}
\tightlist
\item
  \textbf{Logistic Regression}
\end{enumerate}

\begin{verbatim}
LogisticRegression(max_iter=1000)
\end{verbatim}

\begin{enumerate}
\def\labelenumi{\arabic{enumi}.}
\setcounter{enumi}{1}
\tightlist
\item
  \textbf{Random Forest}
\end{enumerate}

\begin{verbatim}
RandomForestClassifier(class_weight='balanced', n_estimators=300,
                       random_state=42)
\end{verbatim}

\begin{enumerate}
\def\labelenumi{\arabic{enumi}.}
\setcounter{enumi}{2}
\tightlist
\item
  \textbf{RNN}
\end{enumerate}

\begin{verbatim}
<keras.src.callbacks.history.History object at 0x17fe6bee0>
\end{verbatim}

\begin{verbatim}
   1/1923 ━━━━━━━━━━━━━━━━━━━━ 46s 24ms/step 209/1923 ━━━━━━━━━━━━━━━━━━━━ 0s 241us/step 428/1923 ━━━━━━━━━━━━━━━━━━━━ 0s 235us/step 647/1923 ━━━━━━━━━━━━━━━━━━━━ 0s 233us/step 868/1923 ━━━━━━━━━━━━━━━━━━━━ 0s 231us/step1089/1923 ━━━━━━━━━━━━━━━━━━━━ 0s 231us/step1310/1923 ━━━━━━━━━━━━━━━━━━━━ 0s 230us/step1533/1923 ━━━━━━━━━━━━━━━━━━━━ 0s 229us/step1752/1923 ━━━━━━━━━━━━━━━━━━━━ 0s 229us/step1923/1923 ━━━━━━━━━━━━━━━━━━━━ 0s 237us/step1923/1923 ━━━━━━━━━━━━━━━━━━━━ 0s 240us/step
\end{verbatim}

\section{Results}\label{results}

\section{Conclusion}\label{conclusion}

\section{References}\label{references}

\begin{itemize}
\item
  Pytlak, K. (n.d.). Personal key indicators of heart disease {[}Data
  set{]}. Kaggle.
  \url{https://www.kaggle.com/datasets/kamilpytlak/personal-key-indicators-of-heart-disease/data}
\item
  Centers for Disease Control and Prevention. (2024, December 2). Heart
  disease risk factors.
  \url{https://www.cdc.gov/heart-disease/risk-factors/?CDC_AAref_Val=https://www.cdc.gov/heartdisease/risk_factors.htm}
\end{itemize}

\end{document}
